\documentclass[./main.tex]{subfiles}

\begin{document}

\section{Ejercicio 3: Programación Dinámica}
\label{sec:ej3}

\subsection{Presentación}
\label{sec:ej3-intro}

\paragraph{} El tercer ejercicio nos pide trabajar con una variante del \textbf{Ejercicio \ref{sec:ej2}}.

\paragraph{} Se trabaja sobre el mismo contexto, con un cuarto de largo \(l\) y ancho \(w\), y \(n\) aspersores de radio \(r_i\) y posición \((x_i, \frac{w}{2})\), con \(0 \leq x_i \leq l \in \mathbb{N}_0\). Pero ahora cada aspersor también tiene un costo \(c_i \in \mathbb{N}\). Y se busca cubrir todo el rectángulo sumando la menor cantidad de costo posible. Devolviendo el costo mínimo o \(-1\) si no se puede cubrir el rectángulo entero.

\subsection{Algoritmo}
\label{sec:ej3-algo}

\paragraph{} Como es un problema de combinatoria utilizamos \textbf{Backtracking} para resolverlo. \\
Reducimos el problema a decidir qué aspersor conviene poner luego de haber puesto cualquier otro aspersor. \\
Expresamos el problema con la siguiente función, donde los parámetros \(i\) y \(j\) corresponden al \(i\)-esimo y \(j\)-esimo aspersor respectivamente. Y además, \(i > j\).
\begin{equation}
  f(i, j) = \begin{cases}
     0  & \text{si } estaLleno(j) \\
     +\infty & \text{si } i > n \lor_L \neg puedeLlenarlo(i, j) \\
     min((f(i+1, j), f(i+1, i) + costo_i)) & otherwise
   \end{cases}
\end{equation}

\begin{equation}
 estaLleno(j) = (l = 0) \lor (i \neq 0 \land_L der_j \geq l)
\end{equation}

\begin{equation}
 puedeLlenarlo(i, j) = (j = 0 \land izq_i \leq 0) \lor_L izq_i \leq der_j
\end{equation}

\paragraph{} El problema se resuelve con \(f(1, 0)\) ya que devuelve el mínimo costo de un conjunto de aspersores que pertenece a todos aquellos conjuntos de aspersores que riegan completamente el terreno. Es decir, si \(\mathbb{O} = \{o_1, \ldots, o_m\}\) la solución óptima, siendo \(\mathbb{S}\) el conjunto de soluciones posibles, entonces se tiene \(\mathbb{O} \in \mathbb{S} \iff f(1, 0) = \sum_{o \in \mathbb{O}}costo_o\). \\
Esta función tiene este comportamiento ya que existen dos casos, uno en el que el aspersor actual \(i\) forma parte de una solución parcial \(\mathbb{O}'\) de \(k\) elementos, y otro donde no lo hace. Si forma parte, la solución total al problema será \(costo_i + \sum_{o' \in \mathbb{O}'}costo_{o'}\). Mientras que si no lo hace, entonces el costo de la solución óptima \(\sum_{o' \in \mathbb{O}'}costo_{o'}\). Siempre y cuando exista ese óptimo, lo elegiremos, mientras que si no hay solución al problema, se devolverá \(+\infty\).

\paragraph{} Como se puede ver, hay \(\bigO{n^2}\) instancias del problema, y para cada una se hacen dos llamadas recursivas, por lo que se llama \(\Omega(2^n)\) veces al problema. Entonces, resaltando que \(n^2 \ll 2^n\), podemos afirmar que hay \textbf{superposición de problemas} y nos conviene usar \textbf{Programación Dinámica}. \\
Empleamos un enfoque \textbf{Bottom-Up} en el que llenamos iterativamente una matrix \(M \in \mathbb{N}_0^{n \times (n+1)}\) de izquierda a derecha y de abajo hacia arriba, de forma tal que \(M[i][j] = minCost(i, j)\). Y como no se pide reconstruir la solución y en las llamadas de la función siempre tenemos que \(i \leq j\), para ninguna instancia se revisan ni filas más arriba ni columnas más a izquierda, entonces podemos aprovechar y reducir la matriz a un sólo vector que se reuitiliza, por lo que el algoritmo queda con complejidad temporal de \(\bigO{n^2}\) y espacial de \(\bigO{n}\).

\end{document}
