\documentclass[a4paper]{article}
\usepackage{biblatex} % Para la bibliografía.
\addbibresource{bibliografia.bib} % Archivo de bibliografía que usa 'biblatex'.
\usepackage{amsfonts} % Para algunos símbolos matemáticos.
\usepackage{hyperref} % Para hipervínculos clickeables.
\input{Algo1Macros}
\usepackage[ruled,spanish,nofillcomment]{algorithm2e} % Para escribir algoritmos.
\usepackage{caption} % Para centrado de captions (pie) de las figuras.
\usepackage{algorithm}
\usepackage{algorithmic}

\usepackage{a4wide}
\usepackage{amsmath, amscd, amssymb, amsthm, latexsym}
\usepackage[utf8]{inputenc}
\usepackage[spanish, activeacute]{babel}
\usepackage{enumerate}
\usepackage{graphicx}
\usepackage{float}
\usepackage{wrapfig}

\setlength{\parskip}{0.1em}
\usepackage{caratula} % Version modificada para usar las macros de algo1 de ~> https://github.com/bcardiff/dc-tex

\newcommand{\comment}[1]{} % Comando para poder 'comentar' (esconder) bloques largos de texto.

\begin{document}
\titulo{Trabajo Práctico I}
\subtitulo{Técnicas Algorítmicas}
\fecha{21 de septiembre de 2022}
\materia{Algoritmos y Estructuras de Datos III}
\integrante{Cappella Lewi, Federico Galileo}{653/20}{galileocapp@gmail.com}
\integrante{Mallol, Martín Federico Alejandro}{208/20}{martinmallolcc@gmail.com}
\integrante{Teplizky, Gonzalo Hernán}{201/20}{gonza.tepl@gmail.com}
\integrante{Stemberg, Uriel Nicolás}{213/20}{uri.stemberg@gmail.com}
\maketitle
\newpage

\begin{abstract}
    
    Para este trabajo el enfoque ser\'a puesto en experimentar con distintas t\'ecnicas algoritmicas recursivas, con el fin de encontrar algoritmos m\'as r\'apidos y eficientes que aquellos que emplean fuerza bruta. Mas all\'a de que en la mayor\'ia de los casos sea imposible evitar una complejidad exponencial, se intentar\'a acotar lo m\'as posible el tiempo de c\'omputo de cada programa implementado en este trabajo. Se deben resolver tres ejercicios donde, cada uno de ellos se tratar\'a sobre el comportamiento de una de estas tres t\'ecnicas en espec\'ifico: backtracking, algoritmo goloso/greedy/miope (cualquiera de estas tres definiciones es v\'alida), y por \'ultimo, programaci\'on din\'amica. 
    
    En cuanto a los primeros dos ejercicios de este trabajo, el desempeño de las implementaciones ser\'a testeada por un  \textit{juez online}\footnote{\url{https://onlinejudge.org/}}.
    
    El informe se divide entre los tres ejercicios. La estructura de cada ejercicio es la siguiente:
    
    \begin{itemize}
        \item \textbf{Introducción}: Se da un vistazo inicial del problema a resolver y la t\'ecnica algor\'itmica a utilizar. Luego se explica brevemente que caminos se tomaron y que experimentaci\'on se llev\'o a cabo (y con qu\'e fines).
        
        \item \textbf{Desarrollo}: Se explica con qué herramientas fue realizado el ejercicio en particular y de qué trata cada archivo donde fue realizada la implementación de los algoritmos.
        
        \item \textbf{Experimentación, resultados y discusión}: Se visitan las t\'ecnicas de backtracking, algoritmos greedy, y programaci\'on din\'amica. Se da un pantallazo inicial sobre c\'omo es el rendimiento de una implementaci\'on sin podas, con podas (tanto de restricci\'on como de optimalidad), y con memoizaci\'on en los casos donde se haya inclu\'ido. 

        Tambi\'en se incluyen aquellas hip\'otesis que fueron confirmadas, o rechazadas, por los resultados que arrojaron los programas, y su consecuencia en la discusi\'on que conllev\'o en el grupo. Hubo pasos en falso como tambien aciertos. No siempre la implementaci\'on de una poda o la memoizacion con cierta estructura tiene por qu\'e mejorar dr\'asticamente el rendimiento de un algoritmo. 
               
        Luego de conocer cómo se comportan las distintas variantes de los algoritmos implementados en cada ejercicio,  se pone el foco en los tiempos de computos que dichas variantes arrojan y las comparamos entre s\'i para convalidar ciertas suposiciones. 
        
        \item \textbf{Conclusiones}: Se concluye que...
        
        Los valores ideales para los métodos fueron los siguientes: \newline
        \textit{Ejercicio 1}: \textbf{blablabla.} \newline
        \textit{Ejercicio 2}: \textbf{blablabla.} \newline
        \textit{Ejercicio 3}: \textbf{blablabla.}  \newline
    \end{itemize}
    
    \textbf{Palabras clave}: \textit{Fuerza Bruta}, \textit{Backtracking}, \textit{Greedy}, \textit{Programaci\'on Din\'amica}, \textit{Complejidad Temporal}, \textit{Complejidad Espacial}.
\end{abstract}


%A partir de aca se puede empezar a escribir o anexar un nuevo documento%

%\section*{Resumen}
%\input{resumen}

\tableofcontents
\thispagestyle{empty}

\newpage

\section{Ejercicio 1}\label{sec:ej1}
%&pdflatex
\documentclass[./main.tex]{subfiles}

\begin{document}

\section{Ejercicio 1: Robots On Ice}
\label{sec:ej1}

\subsection{Presentación}
\label{sec:ej1-intro}

\paragraph{} La primer consigna plantea resolver el problema \textit{UVa 1098, Robots On Ice}\footnote{\url{https://onlinejudge.org/index.php?option=onlinejudge&Itemid=8&page=show_problem&problem=3539}}. En el que se busca contar todos los caminos posibles dentro de un mapa que cumplan ciertas condiciones.

\paragraph{} El mapa es una grilla rectangular de tamaño \(n \times m\), con \(2 \leq n, m \leq 8 \in \mathbb{N}\). Y los caminos empiezan en la posición \((0, 0)\) y terminan en la posición \((0, 1)\), teniendo que pasar en orden por tres posiciones, o check-ins, en momentos equidistantes de el camino. \\
Tanto \(n, m\), y los check-ins, son parámetros de entrada que el programa lee del standard input. Y resultado se devuelve por el standard output. \\
\indent Dentro del mapa sólo es posible moverse en las cuatro direcciones cardinales, sin poder repetir posiciones previamente pisadas.

\subsection{Algoritmo}
\label{sec:ej1-algo}

\paragraph{} Para resolver este problema utilizamos \textbf{backtracking}, armando a fuerza bruta todos los caminos posibles, descartando los que no cumplan con las condiciones pedidas, y contando los que sí. \\
\indent El algoritmo empieza en la posición inicial dada, y prueba moverse en cada dirección. Si el movimiento fue legal, cumple con las restricciones, y no rompe las restricciones a futuro (ver \textbf{\ref{sec:ej1-podas} Podas}), entonces continúa recursivamente, ahora probando moverse desde esta nueva posición. \\
\indent Como para cada posición se prueban cuatro movimientos, la complejidad temporal es \(\bigO{4^{n*m}}\).

\subsubsection{Podas}
\label{sec:ej1-podas}

\paragraph{} Como parte del algoritmo, implementamos dos tipos de podas para descartar caminos:
\subparagraph{Legales (o de Factibilidad)}
\begin{itemize}
  \item El movimiento no se sale del mapa. Osea, se mueve a una posición \((i, j)\) con alguna de las dos coordenadas fuera de rango).
  \item No se pasó ya por el casillero de destino. Para esto llevamos registro de los casilleros por los que se pasó en una matriz de \(n \times m\) booleanos.
\end{itemize}

\subparagraph{Restrictivas y Preventivas (o de Optimalidad)}
\begin{itemize}
  \item El casillero de destino. Como los check-ins tienen que ser visitados en tiempos equidistantes necesitamos que en el paso \(\dfrac{n*m*i}{4}\) estemos en el \(i\)-ésimo check-in (para \(0 \leq i \leq 4 \in \mathbb{N}_0\), contando a las posiciones \((0, 0)\), y \((0, 1)\) como check-in 0 y 4 respectivamente).
  \item Como sólo nos podemos mover de a un casillero por paso y sólo en las 4 direcciones, nos queda que entre las posiciones \((i, j)\) y \((l, k)\) se tienen mínimo \(|l-i| + |k-j|\) pasos, que es \textit{distancia Manhattan}. Por lo que nos fijamos que estemos a menos de \(|l-i| + |k-j|\) pasos del paso necesario para el check-in \((k, j)\) (con \((i, j)\) la posición actual).
  \item Como no podemos pasar dos veces por el mismo casillero, nos fijamos de no cortar en dos partes desconectadas al mapa. Esto sucedería cuando nos podemos mover tanto a izquierda como a derecha, pero no hacia arriba o hacia abajo, y simétrico pudiendo moverse hacia arriba y hacia abajo, pero no a derecha o izquierda (ver Fig. \ref{fig:ej1-mitad}).
\end{itemize}

\begin{figure}[H]
\centering

\begin{tikzpicture}
  \draw[step=1cm,gray,very thin] (0,0) grid (4,4);
  \draw (0.5, 0.5) node{1};
  \draw (0.5, 1.5) node{2};
  \draw (1.5, 1.5) node{3};
  \draw (2.5, 1.5) node{4};
  \draw[green] (3.5, 1.5) node{5};
  \draw[red] (1, 0) rectangle (4, 1);
  \draw[blue] (0, 2) rectangle (4, 4);
\end{tikzpicture}

\caption{En esta figura se ve un camino posible en un mapa de \(4 \times 4\), y las dos areas desconectadas del mapa marcadas en rojo y en azul.}
\label{fig:ej1-mitad}
\end{figure}

\end{document}


\newpage

\section{Ejercicio 2}\label{sec:ej2}
\documentclass[./main.tex]{subfiles}

\begin{document}

\section{Ejercicio 2: Watering Grass}
\label{sec:ej2}

\subsection{Presentación}
\label{sec:ej2-intro}

\paragraph{} El segundo ejercicio presenta el problema \textit{UVa 10382, Watering Grass}\footnote{\url{TODO: URL}}. Que es un problema de combinatoria y minimización, buscando usar la mínima cantidad de aspersores para cubrir completamente un cuarto.

\paragraph{} El cuarto está definido como un rectángulo de largo \(l\) y ancho \(w\), con \(l, w \in \mathbb{N}_0\), y \(n\) aspersores, cada uno cubriendo un círculo de radio \(r_i\) desde la posición \((x_i, \frac{w}{2})\), con \(0 \leq x_i \leq l \in \mathbb{N}_0\). \\
\(l, w, n\), y cada \(r_i, x_i\) son datos que entran por el standard input. Y el resultado se devuelve por el standard output.

\subsection{Algoritmo}
\label{sec:ej2-algo}

\paragraph{} El problema requiere calcular el area del rectángulo que cada aspersor cubre. Pero se puede ver que de cada aspersor sólo nos importa el area rectangular entre los puntos donde interseca con el rectángulo del cuarto, ya que el area cubierta afuera sólo puede ser ocupada por el area rectangular de otro aspersor o dejaría un area sin cubrir por ningún aspersor (ver Figura \ref{fig:two-sprinlers}). \\
\indent Entonces los aspersores con radio \(r_i < \frac{w}{2}\) pueden ser ignorados, ya que no llegan a intersecar con el lado del rectángulo. Y para los aspersores que quedan se los considera no como círculos, sino como su rectángulo subyacente (ver Figura \ref{fig:simple-sprinkle}).

%TODO: Figura rectángulo y círculos

\paragraph{} Partiendo de esa simplificación armamos un algoritmo goloso que primero ordena los aspersores según menor \(x_{il}\). Y luego busca el \(i_0\) de izquierda a derecha que tenga \(x_{i_0l} \leq 0\) y que maximice la distancia entre 0 y \(x_{i_0r}\), para después repetir este paso, pero buscando \(x_{i_1l} \leq x_{i_0r}\), y máxima distancia entre \(x_{i_0r}\) y \(x_{i_1r}\), y así hasta o mo haber más aspersores (caso que no se puede resolver) o haber llenado el cuarto (caso que se encontró la solución optima, ver \textbf{\ref{sec:ej2-dem} Demostración}). \\
En esencia, repitiendo el problema habiendo avanzado hacia cubrir todo el cuarto.

\paragraph{} Como miramos cada aspersor una vez, el algoritmo resuelve el problema en \(\bigO{n}\) pasos, aunque tiene el overhead \(\bigO{n\ log(n)}\) de ordenar los aspersores. Por lo que el problema lo resolvemos en \(\bigO{n\ log(n)}\) pasos. %TODO: Complejidad de cálculos

\subsection{Demostración}
\label{sec:ej2-dem}

%TODO: Demostración

Con esto, planteamos la estructura de cada \textbf{subproblema} y \textbf{decisión golosa} \textit{i} a resolver: \newline

-\textit{Subproblema i}: Hallar la mínima cantidad de aspersores requeridos con el que podamos cubrir el  rectángulo de forma tal que lo hagamos desde el extremo \textit{i} (visto horizontalmente) hasta el final del rectángulo.\newline

-\textit{Decisión golosa i}: De todos los aspersores cuyo límite izquierdo sea menor o igual a \textit{i}, es decir, que empiecen en \textit{i}, que es hasta donde tenemos cubierto, o antes, tomamos el aspersor que sea de máximo cubrimiento, lo cual según definimos será aquel que maximice \textbf{limiteDer(aspersor) - limiteIzq(aspersor)}. \newline

Nos queda, por último, probar la correctitud del algoritmo. Por la forma en la que trabajamos con este tipo de ejercicios, donde el peso se pone en gran parte sobre la demostración, nos propusimos a ver que: \newline

$\bullet$ Si tengo una solución optima puedo modificarla para que use una elección golosa (1). \newline

$\bullet$ Si tengo una secuencia de k decisiones golosas, puedo extenderlas para llegar a una óptima (2).\newline

Lo que queremos con esto, es probar que el algoritmo goloso propuesto produce una solución óptima. \newline \newline 

Comenzamos probando \textbf{(1)}: Queremos ver que \textbf{toda solución óptima para este problema es posible modificarla utilizando elecciones golosas}. Sabemos que existe la óptima, pero queremos aquella que use la golosa. Esto lo podemos demostrar de forma directa. \newline 

Tomamos \(R_{k} = r_{1}, \ldots, r_{k}\) una solución óptima del subproblema \textit{i}, es decir, aquella que minimiza la cantidad de aspersores necesarios para cubrir desde el punto i hasta el final del rectángulo, y un \textit{r} \(\in R_{k}\) tal que \textit{r} representa un intervalo que contiene al \textit{i}. Seguro que existe un \textit{r} de estas características ya que no sería óptima la solución si hubiera algún pedazo del rectángulo no cubierto. \newline

Como también habíamos mencionado, una decisión golosa en ese punto para resolver el subproblema \textit{i}, será tomar el aspersor de mayor extensión que cubra desde \textit{i} o antes al rectángulo.
Si llamamos a ese aspersor \(G_{i}\), y nuestro objetivo es poder introducirlo dentro de la solución óptima, lo que podemos hacer es, partiendo de \(R_{k}\), crear \(R_{k}' = R_{k} \cup G_{i} - r\). \newline 

Sabiendo que el aspersor $G_{i}$ es el de máximo cubrimiento que pasa por \textit{i}, entonces seguro que su cubrimiento sera $\geq$ al cubrimiento de \textit{r}, además de que, considerando que lo que hicimos fue básicamente reemplazar un aspersor por otro, seguiremos usando la mínima cantidad posible y cubriendo todo el rectángulo. Por lo tanto, $ R_{k}'$ es óptima utilizando una elección golosa, y en general, podremos fabricarnos todas las $ R_{k}''$ que deseemos reemplazando en cada subintervalo por la decisión golosa en ese punto, manteniéndonos en una solución óptima.\done\newline\newline

Por último, probemos \textbf{(2)}: queremos ver que \textbf{luego de tomar \textit{k} decisiones golosas $G_{k}$, $\forall k > 0$, se puede extender hacia una solución óptima.} Veámoslo por inducción en \textit{k}. \newline 

P(k): $G_{k}$ se puede extender a una solución óptima $\forall k > 0$. \newline

\textbf{Caso base con k = 0}: Como $G_{0}$ = $\emptyset$, es decir, aún no hemos tomado ninguna decisión golosa, luego si existe una solución óptima, lo voy a poder extender a ella.  \newline

\textbf{Paso inductivo}: queremos ver que si vale P(k), lo hace también P(k+1). 
Nuestra HI es que tomadas \textit{k} decisiones golosas, podemos extendernos hacia una sol. óptima, y queremos saber si vale para k+1 decisiones golosas. \newline

Sabemos que a medida que vamos tomando estas decisiones, el subproblema restante se hace cada vez más chico. Tomadas k elecciones golosas, y con el rectángulo cubierto hasta por ejemplo, \textit{k}, nos queda por resolver el subproblema \textit{k+1}, que encuentre los aspersores mínimos necesarios para cubrir el pedazo de rectángulo que se extiende desde k hasta el final del mismo. \newline

Si \textbf{por HI} sabemos que existe un $\gamma$ tal que \(G_{1} \cup \ldots \cup G_{k} \cup \gamma\) cubren todo el rectángulo, luego $\gamma$ es óptima para el subproblema \textit{k+1}. Pero adicionalmente, por lo demostrado en \textbf{(1)}, toda solución óptima puede modificarse con elecciones golosas. \newline

Por lo tanto, con el mismo cubrimiento y la misma cantidad de aspersores utilizados, tendremos también una solución óptima que utilice la decisión golosa  $G_{k+1}$. Esto quiere decir que habrá un $ \gamma'$ que asegure que \(G_{1} \cup \ldots \cup G_{k} \cup G_{k+1} \cup \gamma'\) cubren todo el rectángulo usando la mínima cantidad de aspersores y por lo tanto, $ \gamma'$ es óptima para el subproblema \textit{k+2}, probando así que $G_{k+1}$ se puede también extender a una solución óptima, \textbf{como queríamos ver}.\done\newline\newline

\end{document}


\newpage

\section{Ejercicio 3}\label{sec:ej3}
\documentclass[./main.tex]{subfiles}

\begin{document}

\section{Ejercicio 3: Programación Dinámica}
\label{sec:ej3}

\subsection{Presentación}
\label{sec:ej3-intro}

\paragraph{} El tercer ejercicio nos pide trabajar con una variante del \textbf{Ejercicio \ref{sec:ej2}}.

\paragraph{} Se trabaja sobre el mismo contexto, con un cuarto de largo \(l\) y ancho \(w\), y \(n\) aspersores de radio \(r_i\) y posición \((x_i, \frac{w}{2})\), con \(0 \leq x_i \leq l \in \mathbb{N}_0\). Pero ahora cada aspersor también tiene un costo \(c_i \in \mathbb{N}\). Y se busca cubrir todo el rectángulo sumando la menor cantidad de costo posible. Devolviendo el costo mínimo o \(-1\) si no se puede cubrir el rectángulo entero.

\subsection{Algoritmo}
\label{sec:ej3-algo}

\paragraph{} Como es un problema de combinatoria utilizamos \textbf{Backtracking} para resolverlo. \\
Reducimos el problema a decidir qué aspersor conviene poner luego de haber puesto cualquier otro aspersor. \\
Expresamos el problema con la siguiente función, donde los parámetros \(i\) y \(j\) corresponden al \(i\)-esimo y \(j\)-esimo aspersor respectivamente. Y además, \(i > j\).
\begin{equation}
  f(i, j) = \begin{cases}
     0  & \text{si } estaLleno(j) \\
     +\infty & \text{si } i > n \lor_L \neg puedeLlenarlo(i, j) \\
     min((f(i+1, j), f(i+1, i) + costo_i)) & otherwise
   \end{cases}
\end{equation}

\begin{equation}
 estaLleno(j) = (l = 0) \lor (i \neq 0 \land_L der_j \geq l)
\end{equation}

\begin{equation}
 puedeLlenarlo(i, j) = (j = 0 \land izq_i \leq 0) \lor_L izq_i \leq der_j
\end{equation}

\paragraph{} El problema se resuelve con \(f(1, 0)\) ya que devuelve el mínimo costo de un conjunto de aspersores que pertenece a todos aquellos conjuntos de aspersores que riegan completamente el terreno. Es decir, si \(\mathbb{O} = \{o_1, \ldots, o_m\}\) la solución óptima, siendo \(\mathbb{S}\) el conjunto de soluciones posibles, entonces se tiene \(\mathbb{O} \in \mathbb{S} \iff f(1, 0) = \sum_{o \in \mathbb{O}}costo_o\). \\
Esta función tiene este comportamiento ya que existen dos casos, uno en el que el aspersor actual \(i\) forma parte de una solución parcial \(\mathbb{O}'\) de \(k\) elementos, y otro donde no lo hace. Si forma parte, la solución total al problema será \(costo_i + \sum_{o' \in \mathbb{O}'}costo_{o'}\). Mientras que si no lo hace, entonces el costo de la solución óptima \(\sum_{o' \in \mathbb{O}'}costo_{o'}\). Siempre y cuando exista ese óptimo, lo elegiremos, mientras que si no hay solución al problema, se devolverá \(+\infty\).

\paragraph{} Como se puede ver, hay \(\bigO{n^2}\) instancias del problema, y para cada una se hacen dos llamadas recursivas, por lo que se llama \(\Omega(2^n)\) veces al problema. Entonces, resaltando que \(n^2 \ll 2^n\), podemos afirmar que hay \textbf{superposición de problemas} y nos conviene usar \textbf{Programación Dinámica}. \\
Empleamos un enfoque \textbf{Bottom-Up} en el que llenamos iterativamente una matrix \(M \in \mathbb{N}_0^{n \times (n+1)}\) de izquierda a derecha y de abajo hacia arriba, de forma tal que \(M[i][j] = minCost(i, j)\). Y como no se pide reconstruir la solución y en las llamadas de la función siempre tenemos que \(i \leq j\), para ninguna instancia se revisan ni filas más arriba ni columnas más a izquierda, entonces podemos aprovechar y reducir la matriz a un sólo vector que se reuitiliza, por lo que el algoritmo queda con complejidad temporal de \(\bigO{n^2}\) y espacial de \(\bigO{n}\).

\end{document}



\newpage


\begin{thebibliography}{1}

\bibitem{blabla} https://www.blablabla.com/blabla/
\bibitem{blabla} https://www.blablabla.com/blabla/
\bibitem{blabla} https://www.blablabla.com/blabla/

\end{thebibliography}

\end{document}
